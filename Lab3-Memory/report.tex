\documentclass[dvipsnames, svgnames, x11names, a4paper, 11pt]{article}

% URLs and hyperlinks ------------------------------
\usepackage{hyperref}
\hypersetup{
    colorlinks=true,
    linkcolor=NavyBlue,
    filecolor=magenta,      
    urlcolor=blue,
}
\usepackage{xurl}
%---------------------------------------------------

\usepackage{graphicx}
\usepackage{listings}
\usepackage{color}
\usepackage{xcolor}

\definecolor{dkgreen}{rgb}{0,0.6,0}
\definecolor{gray}{rgb}{0.5,0.5,0.5}
\definecolor{mauve}{rgb}{0.58,0,0.82}

\lstset{frame=tb,
    language=vhdl,
    aboveskip=3mm,
    belowskip=3mm,
    showstringspaces=false,
    columns=flexible,
    basicstyle=\ttfamily,
    numbers=left,
    numberstyle=\small\color{gray},
    keywordstyle=\bfseries\color{Green4},
    commentstyle=\color{gray},
    stringstyle=\color{mauve},
    breaklines=true,
    breakatwhitespace=true,
    tabsize=4,
    identifierstyle=\color{black}
}

\usepackage{xepersian}
\settextfont{Yas}

\title{حافظه‌های \lr{ROM} و \lr{BRAM}}
\author{مهدی حق‌وردی}

\begin{document}
\maketitle
\tableofcontents

\section{توضیحات فایل‌های تمرین}
\subsection{فایل \lr{\texttt{rom8x8}}}
این فایل، ساده‌ترین فایل این تمرین است که شامل یک موجودیت به نام
\lr{\texttt{rom8x8}}
و یک معماری به نام 
\lr{\texttt{behave}}
است.

\subsubsection{توضیحات موجودیت \lr{\texttt{rom8x8}}}
این موجودیت شامل دو پورت است:
\begin{itemize}
\item 
\lr{\texttt{addr}}

این پورت یک ورودی 
\lr{\texttt{std\_logic\_vector}} 
۳ بیتی است که در واقع مقدار خروجی را تعیین می‌کند.

\item 
\lr{\texttt{dout}}

این پورت هم یک پورت خروجی از نوع 
\lr{\texttt{std\_logic\_vector}}
۷ بیتی است که مقدار خروجی روی آن قرار می‌گیرد.
\end{itemize}

\subsubsection{توضیحات معماری \lr{\texttt{behave}}}
این تنها معماری حاضر برای موجودیت 
\lr{\texttt{rom8x8}}
است که در واقع یک \lr{decoder} ۳ به ۸ است.

مقادیری که برای خروجی انتخاب شده‌اند کاملا \lr{hard code} شده‌اند و این کاملا با نام موجودیت (که \lr{\texttt{rom}} است) هماهنگی دارد.

در معماری از 
\lr{language feature}
عی به نام 
\lr{\texttt{with ... select}}
استفاده شده است که با توجه به مقدار 
\lr{\texttt{addr}}
خروجی را روی 
\lr{\texttt{dout}}
قرار می‌دهد.
\subsection{فایل \lr{romram}}

\subsection{فایل \lr{clockDevider}}

\end{document}